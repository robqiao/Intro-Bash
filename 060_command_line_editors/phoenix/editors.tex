\setModuleTitle{Command Line Editors}
\setModuleAuthors{%
  Robert Qiao, Research Services Phoenix Team, University of Adelaide
\mailto{robert.qiao@adelaide.edu.au}\\ 
}

%----------------------------------------------------------------------------------------
% MODULE TITLE PAGE
%----------------------------------------------------------------------------------------
% BEGIN: Module Title Page
%  * The chapter page will always appear on odd numbered page
\chapter{\moduleTitle}
\newpage

In addition to the streaming editor named sed as you have studied earlier, which operates on a file
in a non-interactive mode according to a set of instructions that you specify on the command line or
in a
script. It can be used for "canned" editing tasks that must be applied in the same way to many
files. More commonly, we use interactive editors and there are three powerful interactive text
editors that
are commonly used on Unix computers: vi, pico, and emacs. \\

Although vi, pico, emacs are extremely powerful, to get a firm grasp takes some efforts and most of
times we just need to open, amend and save a file without remembering the keyboard shortcuts.
Luckly, there is such editor called nano and that is what we will use for the class today.
The nano editor has its own set of keyboard shortcuts of course and in this guide I aim to help you
to understand the meaning of all those special keystrokes you can use to make your life easier when
using nano. \\
But before we dive into the nano, let's generally go over some of the streamline editors like vi,
emacs. 

\section {vi}

\begin{information}
Vi (pronounced "vee-eye") is the standard editor for Unix systems. It is universally available on
Unix systems. Vi is a screen editor. It treats your computer screen as a window into the file. You
move the window around to view different parts of the file. You move the cursor to the location on
the screen where you want to make a change; or optionally, you specify some kind of global change.
Vi updates your screen to reflect changes that you make in the file. Actually, it works on a copy of
the file in memory, and only updates the file on disk when you tell it to, such as when you end the
editing session. \\
Main advantages of vi are including: 
\begin{itemize}
  \item Vi is universally available on Unix systems. It has been around so long in a stable form
that it is essentially bug free. Many clones have been written for other kinds of computers.
  \item Vi has many powerful commands that utilize just the alphanumeric keys -- it does not require
special function keys.
  \item Vi is a small program that does not require a lot of system memory or CPU time. It works
very fast, even on large files.
  \item While vi is not programmable, it has a simple way to let other Unix programs, such as the
sort utility, work on selected portions of your file. This adds the functionality of all those
programs to the editor.
  \item Vi is completely terminal device independent. It will work with any kind of terminal.
A system file describes the capabilities and control sequences of each kind of terminal for vi. All
the program needs to know is what type of terminal you have. When you log in, if pangea cannot
figure out what kind of terminal you have, it will prompt you to specify a terminal type. The most
common type is the vt100, which most modern terminals and PC communications software emulate.
\end{itemize}
\end{information} 
